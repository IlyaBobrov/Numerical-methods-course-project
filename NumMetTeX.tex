\documentclass[14pt, titlepage, a4paper]{extarticle} %comment
\usepackage{fancyvrb}
\usepackage[utf8]{inputenc}
\usepackage[T2A]{fontenc}
\usepackage[russian]{babel}
\usepackage{amssymb, amsfonts}
\usepackage{amsmath}

\usepackage{graphicx}

\usepackage{lipsum}
\usepackage{style}
\textwidth 16cm
\textheight 25cm
\oddsidemargin -2.5mm
\evensidemargin -3mm
\topmargin -20mm
\parindent 1.25cm

\usepackage{comment}
%\usepackage{autonum}


\begin{document}
	
	%%%%%%%%%%%%%%%%%%%%PAGE%1%%%%%%%%%%%%%%%%%%%%
	%%%%%%%%%%%%%%%%%%%Вкладка%%%%%%%%%%%%%%%%%%%%
	
	\begin{center}
		ЗАДАНИЕ НА КУРСОВОЙ ПРОЕКТ\\
		по дисциплине «Численные методы»
	\end{center}
	
	~\\	
	Ф.И.О. \uline{\vspace{10pt} Бобров И.И., Игнатьев М.В., Абдуллаев Т.М.~~~~~~~~~~~}\\
	~\\
	~\\
	ТЕМА курсового проекта
	
	~\\
	Рекуррентные формулы и интегрирование по Ромбергу.
	\\
	~\\
	~\\
	ФОРМУЛИРОВКА  задания:
	
	\begin{itemize}
		\item Создать алгоритм решения поставленной задачи, реализовать его, протестировать программы;
		\item Оформить и представить итоги проделанной работы в виде отчета;
		\item Сформулировать выводы по полученным решениям, отметить достоинства и недостатки методов.
	\end{itemize}

	~\\
	РУКОВОДИТЕЛЬ проекта\uline{\hspace{80pt}}/ Пак Т.В./\\	
	\\
	~\\
	ДАТА ВЫДАЧИ задания\uline{~6~ноября~2021~}\\
	\\
	~\\
	СРОК ВЫПОЛНЕНИЯ задания 09.11.2020 – 10.01.2021\\
	\\
	~\\
	Задание получил \uline{Бобров И.И., Игнатьев М.В., Абдуллаев Т.М.}\\
	
	\thispagestyle{empty}
	\newpage
	
	%%%%%%%%%%%%%%%%%%%%PAGE%2%%%%%%%%%%%%%%%%%%%%
	%%%%%%%%%%%%%%%%%%Титульник%%%%%%%%%%%%%%%%%%%
	
	\begin{comment}
	\fefutitlepage{Б8118-01.03.02миопд}{Бобров И.И.}{Игнатьев М.В.}{Абдуллаев Т.М.}{5}
	\defaultfont
	\end{comment}

	%%%%%%%%%%%%%%%%%%%%PAGE%3%%%%%%%%%%%%%%%%%%%%
	%%%%%%%%%%%%%%%%%%Содержание%%%%%%%%%%%%%%%%%%

	\tableofcontents
	\pagebreak
	
	%%%%%%%%%%%%%%%%%%%%PAGE%4%%%%%%%%%%%%%%%%%%%%
	%%%%%%%%%%%%%%%%%%%Введение%%%%%%%%%%%%%%%%%%%
	
	\section*{Рекуррентные формулы и интегрирование по Ромбергу}
	\addcontentsline{toc}{section}{Рекуррентные формулы и интегрирование по Ромбергу.}
	
	\subsection*{Введение}
	\addcontentsline{toc}{subsection}{Введение}
	
	Объектом исследования являются !!!!!!!!.\\
	\textit{Цель работы} – ознакомиться с численными методами анализа и решения !!!!!!!!!, решить предложенные типовые задачи, сформулировать выводы по полученным решениям, отметить достоинства и недостатки методов, приобрести практические навыки и компетенции, а также опыт самостоятельной профессиональной деятельности, а именно:
	\begin{itemize}
		\item создать алгоритм решения поставленной задачи и реализовать его, протестировать программы;
		\item освоить теорию вычислительного эксперимента; современных компьютерных технологий; 
		\item приобрести навыки представления итогов проделанной работы в виде отчета, оформленного в соответствии с имеющимися требованиями, с привлечением современных средств редактирования и печати.	
	\end{itemize}
	Работа над курсовым проектом предполагает выполнение следующих задач:
	\begin{itemize}
		\item дальнейшее углубление теоретических знаний обучающихся и их систематизацию;
		\item получение и развитие прикладных умений и практических навыков по направлению подготовки;
		\item овладение методикой решения конкретных задач;
		\item развитие навыков самостоятельной работы;
		\item развитие навыков обработки полученных результатов, анализа и осмысления их с учетом имеющихся литературных данных;
		\item приобретение навыков оформления описаний программного продукта;
		\item повышение общей и профессиональной эрудиции.
	\end{itemize}

	
	\pagebreak
	
	%%%%%%%%%%%%%%%%%%%%PAGE%5%%%%%%%%%%%%%%%%%%%%
	%%%%%%%%%%%%%%%%Основная%часть%%%%%%%%%%%%%%%%
	
	\subsection*{Основная часть}
	\addcontentsline{toc}{subsection}{Основная часть}
	
	
	\pagebreak
	
	%%%%%%%%%%%%%%%%%%%%PAGE%?%%%%%%%%%%%%%%%%%%%%
	%%%%%%%%%%%%%%%%%%Заключение%%%%%%%%%%%%%%%%%%
	
	\subsection*{Заключение}
	\addcontentsline{toc}{subsection}{Заключение}
	
	В результате работы над курсовым проектом приобрел практические навыки владения:
	\begin{itemize}
		\item современными численными методами решения задач математической физики;
		\item основами алгоритмизации для численного решения задач математической физики на одном из языков программирования;
		\item инструментальными средствами, поддерживающими разработку программного обеспечения для численного решения задач математической физики; 
	\end{itemize}
	а также навыками представления итогов проделанной работы в виде отчета, оформленного в соответствии с имеющимися требованиями, с привлечением современных средств редактирования и печати.
	
	\pagebreak

	%%%%%%%%%%%%%%%%%%%%PAGE%?%%%%%%%%%%%%%%%%%%%%
	%%%%%%%%%%%%%%%%%%%Источники%%%%%%%%%%%%%%%%%%
	
	\subsection*{Список используемых источников}
	\addcontentsline{toc}{subsection}{Источники}
	
	\textbf{Источники}
	\begin{enumerate}
		\item Численные методы. Использование Matlab. Третье издание. Джон Г. Мэтьюз Издательский дом $"$Вильямс$"$ 2001
	\end{enumerate}
	
	\pagebreak
	
	%%%%%%%%%%%%%%%%%%%%PAGE%?%%%%%%%%%%%%%%%%%%%%
	%%%%%%%%%%%%%%%%%%Приложения%%%%%%%%%%%%%%%%%%
	
	\subsection*{Приложения}
	\addcontentsline{toc}{subsection}{Приложения}
	
	\pagebreak	
	
	
\end{document}